\section{Tomasz Strycharz}
Wyrażenie matematyczne
Równanie koła na płaszczyźnie o środku (a,b) i promieniu równym r 
\begin{math} (x-a)^2+(y-b)^2 \leq r^2 \end{math} .

Zdjęcie

\begin{figure}[h]
    \centering
    \includegraphics[scale=1.5]{pictures/żubr.jpg}
    \caption{Zdjęcie żubra.}
\end{figure}


Tabela ~\ref{tab:tabela6}

% Please add the following required packages to your document preamble:
% \usepackage{multirow}
\begin{table}[htbp]
\centering
\begin{tabular}{|c|c|c|c|}
\hline
p & q & p \lor  q & p \land  q \\ \hline
0 & 0 & 0        & 0         \\ \hline
0 & 1 & 1        & 0         \\ \hline
1 & 0 & 1        & 0         \\ \hline
1 & 1 & 1        & 1         \\
\hline 
\end{tabular}
\caption{Tabela wartości logicznych}
\label{tab:tabela6}
\end{table}

Lista numerowane
\begin{enumerate}
  \item Jeden
  \item Dwa
  \item Trzy
\end{enumerate}
5) Lista nienumerowana
\begin{itemize}
  \item a
  \item[!] b
  \item[$\blacksquare$] c
\end{itemize}

Krótki tekst
\begin{flushleft}
\textbf{LaTeX} – \textit{oprogramowanie do zautomatyzowanego składu tekstu,} a także związany z nim język znaczników, służący do formatowania dokumentów tekstowych i tekstowo-graficznych (na przykład: broszur, artykułów, książek, plakatów, prezentacji, a nawet stron HTML).
\end{flushleft}

\begin{center}
\textbf{LaTeX} \underline{nie jest} samodzielnym środowiskiem programistycznym: jest to zestaw makr stanowiących nadbudowę dla systemu składu TEX, automatyzujących czynności związane z procesem składania tekstu. \emph{Jednak, ze względu na dużą popularność LaTeX-a (w porównaniu z czystym TeX-em) nazwy te bywają używane zamiennie.}
\end{center}
https://pl.wikipedia.org/wiki/LaTeX