\section{Patryk Podgórski}

1) Działanie matematyczne 
\begin{equation}
    |AB|=\sqrt{(x_b-x_a)^2+(y_b-y_a)^2}
\end{equation} 
2) zdjęcie kota
\begin{figure}[htbp]
    \centering
    \includegraphics[scale=0.2]{pictures/kot.jpg}
    \caption{kot}
    \label{fig:kot}
\end{figure}

3) Tabela:~\ref{tab:moja_tabela}
\begin{table}[htbp]
\centering
\begin{tabular}{|c|c|c|c|}
\hline
  &      \textbf{A}  & \textbf{B}           &\textbf{C}           \\ \hline
  
\textbf{1} &      A1 &    B1       & C1          \\ \hline

\textbf{2} &      A2 &    B2       & C2          \\ \hline
\end{tabular}
\caption{tabela 3x2}
\label{tab:moja_tabela}
\end{table}

4) lista numerowana
\begin{enumerate}
    \item Aaa
    \item Bbb
    \item Ccc
\end{enumerate}

5) lista nienumerowana
\begin{itemize}
    \item {O.o} pierwszy przedmiot
    \item {+} drugi przedmiot
\end{itemize}

6) tekst

To najprostsze placki na świecie, a przepisu \textbf{nie da się zapomnieć}, bowiem podstawę do odmierzenia dwu podstawowych składników, tj. mąki i mleka, stanowi ten sam (\underline{dowolny}) kubek.

\textit{Jego} rozmiar każdy może dobrać indywidualnie, w zależności od potrzeb. Pomysł na takie proporcje składników podpatrzyłem w jednym ze świątecznych programów \textbf{Jamie Olivera}.

https://www.facetikuchnia.com.pl/najprostszy-przepis-na-placki/